\let\negmedspace\undefined
\let\negthickspace\undefined
\documentclass[journal,12pt,twocolumn]{IEEEtran}
\usepackage{cite}
\usepackage{amsmath,amssymb,amsfonts,amsthm}
\usepackage{algorithmic}
\usepackage{graphicx}
\usepackage{textcomp}
\usepackage{xcolor}
\usepackage{txfonts}
\usepackage{listings}
\usepackage{enumitem}
\usepackage{mathtools}
\usepackage{gensymb}
\usepackage{comment}
\usepackage[breaklinks=true]{hyperref}
\usepackage{tkz-euclide} 
\usepackage{listings}
\usepackage{gvv}                                        
%\def\inputGnumericTable{}                                 
\usepackage[latin1]{inputenc}                                
\usepackage{color} 
\usepackage{array}
\usepackage{longtable}
\usepackage{calc}   
\usepackage{multirow}
\usepackage{hhline}
\usepackage{ifthen}
\usepackage{lscape}
\usepackage{tabularx}
\usepackage{array}
\usepackage{float}

\newtheorem{theorem}{Theorem}[section]
\newtheorem{problem}{Problem}
\newtheorem{proposition}{Proposition}[section]
\newtheorem{lemma}{Lemma}[section]
\newtheorem{corollary}[theorem]{Corollary}
\newtheorem{example}{Example}[section]
\newtheorem{definition}[problem]{Definition}
\newcommand{\BEQA}{\begin{eqnarray}}
\newcommand{\EEQA}{\end{eqnarray}}
\newcommand{\define}{\stackrel{\triangle}{=}}
\theoremstyle{remark}
\newtheorem{rem}{Remark}

% Marks the beginning of the document
\begin{document} 

\bibliographystyle{IEEEtran}
\vspace{3cm}

\title{03-17-2021 SHIFT-1-1-15}

\author{EE24BTECH11029- JANAGANI SHRETHAN REDDY}
\maketitle{}
\newpage
\bigskip
\renewcommand{\thefigure}{\theenumi}
\renewcommand{\thetable}{\theenum}
\begin{enumerate}
    \item Which of the following is true for $y\brak{x}$ that satisfies the differential equation $\brak{\frac{dy}{dx}}=xy-1+x-y;y\brak{0}=0$
    \begin{enumerate}
        \item $y\brak{1}=1$
        \item $y\brak{1}=e^{\frac{1}{2}}-1$
        \item $y\brak{1}=e^{\frac{1}{2}}-e^{\frac{-1}{2}}$
        \item $y\brak{1}=e^{\frac{-1}{2}}-1$\\
    \end{enumerate}
    \item The system of equations $kx+y+z=1, x+ky+z=k$ and $x+y+zk=k^2$ has no solution if $k$ is equal to
    \begin{enumerate}
        \item $-2$
        \item $-1$
        \item $1$
        \item $0$\\
    \end{enumerate}
    \item The value of $4+\frac{1}{5+\frac{1}{4+\frac{1}{5+\frac{1}{4+\dots\infty}}}}$
    \begin{enumerate}
        \item $2+\brak{\frac{4}{\sqrt{5}}}\brak{\sqrt{30}}$
        \item $4+\brak{\frac{4}{\sqrt{5}}}\brak{\sqrt{30}}$
        \item $2+\brak{\frac{2}{5}}\brak{\sqrt{30}}$
        \item $5+\brak{\frac{2}{5}}\brak{\sqrt{30}}$\\
    \end{enumerate}
    \item If the Boolean expression $\brak{p\implies q}\Leftrightarrow\brak{q*\brak{\sim p}}$  is a tautology, then the Boolean expression $p*\brak{\sim q}$ is equivalent to:
    \begin{enumerate}
        \item $p\implies\sim q$
        \item $p\implies q$
        \item $q\implies p$
        \item $\sim q\implies p$\\
    \end{enumerate}
    \item  Choose the incorrect statement about the two circles whose equations are given below: $x^2+y^2-10x-10y+41=0$ and $x^2+y^2-16x-10y+80=0$
    \begin{enumerate}
        \item  Distance between two centres is the average radii of both the circles
        \item Circles have two intersection points
        \item Both circles centres lie inside the region of one another
        \item Both circles pass through the centre of each other\\
    \end{enumerate}
    \item  The sum of possible values of $x$ for $\tan^{-1}\brak{x+1}+\cot^{-1}\brak{\frac{1}{\brak{x-1}}}=\tan^{-1}\brak{\frac{8}{31}}$ is:
    \begin{enumerate}
        \item $-\frac{33}{4}$
        \item $-\frac{32}{4}$
        \item $-\frac{31}{4}$
        \item $-\frac{30}{4}$\\
    \end{enumerate}
    \item Let $\vec{a}=2i-3j+4k,\vec{b}=7i+j-6k.$ If $\vec{r}$X$\vec{a}=\vec{r}$X$\vec{b},\vec{r}.\brak{i+2j+k}=-3,$ then $\vec{r}.\brak{2i-3j+k}=$ is equal to
    \begin{enumerate}
        \item $10$
        \item $13$
        \item $12$
        \item $8$
     \end{enumerate}
    \item The equation of the plane which contains the $y-axis$ and passes through the point $\brak{1, 2, 3}$ is:
    \begin{enumerate}
        \item $3x+z=6$
        \item $3x-z=0$
        \item $x+3z=10$
        \item $x+3z=0$\\
    \end{enumerate}
    \item If $A=\myvec{0&\sin\alpha\\\sin\alpha&0}$ and det$\brak{A^2-\brak{\frac{1}{2}}I}=0,$then a possible value of $a$ is:
   \begin{enumerate}
        \item $\frac{\pi}{6}$
        \item $\frac{\pi}{2}$
        \item $\frac{\pi}{3}$
        \item $\frac{\pi}{4}$\\
    \end{enumerate}
    \item The line $2x-y+1=0$ is a tangent to the circle at the point $\brak{2, 5}$ and the centre of the circle lies on $x-2y=4.$ Then, the radius of the circle is:
    \begin{enumerate}
        \item $4\sqrt{5}$
        \item $3\sqrt{5}$
        \item $5\sqrt{3}$
        \item $5\sqrt{4}$\\
    \end{enumerate}
    \item Team $\Vec{A}$ consists of $7$ boys and $n$ girls and Team $\Vec{B}$ has $4$ boys and $6$ girls. If a total of $52$ single matches can be arranged between these two teams when a boy plays against a boy and a girl plays against a girl, then $n$ is equal to
    \begin{enumerate}
        \item $5$
        \item $6$
        \item $2$
        \item $4$\\
    \end{enumerate}
    \item In a triangle $PQR,$ the coordinates of the points $P$ and $Q$ are $\brak{-2, 4}$ and $\brak{4, -2}$ respectively. If the equation of the perpendicular bisector of $PR$ is $2x-y+2=0,$ then the centre of the circumcircle of the $\triangle PQR$ is:
    \begin{enumerate}
        \item $\brak{-2,-2}$
        \item $\brak{0,2}$
        \item $\brak{-1,0}$
        \item $\brak{1,4}$\\
    \end{enumerate}
    \item If $\cot^{-1}\brak{a}=\cot^{-1}\brak{2}+\cot^{-1}\brak{8}+\cot^{-1}\brak{18}+\cot^{-1}\brak{32}\dots$ upto $100$ terms, then $a$ is:
    \begin{enumerate}
        \item $1.03$
        \item $1.00$
        \item $1.01$
        \item $1.01$\\
    \end{enumerate}
    \item Which of the following statements is incorrect for the function $g\brak{a}$ for $a\in\Vec{R}$ such that $g\brak{a}=\int_{\frac{\pi}{3}}^{\frac{\pi}{6}}\frac{\sin^a{x}}{\brak{\cos^a{x}+\sin^a{x}}}dx$
    \begin{enumerate}
        \item $g\brak{a}$ is a strictly decreasing function
        \item $g\brak{a}$ has an inflexion point a $a=\frac{-1}{2}$
        \item $g\brak{a}$ is an even function
        \item $g\brak{a}$ is a strictly increasing function\\
    \end{enumerate}
    \item If the fourth term in the expansion of $\brak{x+x^{\log_{2}x}}^7$ is $4480,$ then the value of $x$ where $x\in N$ is equal to:
    \begin{enumerate}
        \item $4$
        \item $3$
        \item $2$
        \item $1$
    \end{enumerate}  
\end{enumerate}
\end{document} 
